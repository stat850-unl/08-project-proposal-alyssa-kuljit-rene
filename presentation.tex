\documentclass{beamer} %

%%%BASICS
\usepackage[utf8]{inputenc}
\usepackage{csquotes}
%\usepackage{graphicx} 
\graphicspath{{./Images}}

%%%START THEME SETTINGS
\usetheme{Dresden}
\usecolortheme{beaver}
\usefonttheme{professionalfonts}
\setbeamertemplate{itemize item}{\color{red}$\blacksquare$}
%%%END THEME SETTINGS

%%%START APA
\usepackage[british]{babel}
\usepackage[backend=biber,style=apa]{biblatex}
\DeclareLanguageMapping{british}{british-apa}
\addbibresource{references.bib}
%% APA citing
%% \cite{t} - Uthor und Richter, 2010
%% \textcite{t} - Uthor und Riter (2010)
%% \parencite{t} - (Uthor & Riter, 2010)
%% \parencite[Chapt.~4]{t} - (Uthor & Riter, 2010, S. 15)
%%%END APA


\title[Comparison of Health, Wellness, and Food Accessibility in the USA ]{Comparison of Health, Wellness, and Food Accessibility in the USA }
\institute[UNL]{University of Nebraska-Lincoln}
\author{Kuljit Bhatti, Alyssa Grube, Rene Ingersoll}

\date{\today}

\begin{document}
	
	\begin{frame}
		\titlepage
	\end{frame}
	
	%------------------------------------------------------
	
	
	\begin{frame}{Goals of this Project}
		\begin{itemize}
			\item Compare local rates of obesity with access to recreational centers  
			\item Compare the local rates of food insecurity and low access to food with the number of grocery stores by state  
			\item goal 3 
		\end{itemize}
	\end{frame}
	
	
	\begin{frame}{About the Food Environment Atlas} 
		\begin{itemize}
			\item Track store and restaurant proximity, food prices and assistance programs 
			\item Assembles statistics on food environment indicators 
			\item Stimulate research on the determinants of food choices and diet quality 
			\item Provide a spatial overview of a community's ability to access healthy food 
			\item Track a community's success in accessing healthy food 
		\end{itemize}
	\end{frame}

\begin{frame}{Food Access by State Compared to Grocery Store Availability}
	\includegraphics{grocplot} 

	\end{frame}
	
	\begin{frame}{Food Insecurity by State Compared to Grocery Store Availability} 
		
		Food insecurity decreases with a greater number of grocery stores. 
		
	\end{frame}

\begin{frame}{Comparing State Population Levels with Number of Grocery Stores} 
	
	States with higher populations have a greater number of grocery stores. 
	\end{frame}
	
	\begin{frame}{References} 
	\href{https://www.ers.usda.gov/data-products/food-environment-atlas/go-to-the-atlas/}{Food Environment Atlas }
\end{frame}
	
\end{document}